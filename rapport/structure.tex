%Texte et LuaLaTeX
\usepackage{luatextra}
\usepackage{polyglossia}
\setmainfont{Libertinus Serif}
% \setmainfont{Arial}
\usepackage[locale=FR]{siunitx}
\usepackage{booktabs}
\usepackage{xspace}
\usepackage{pict2e}
\usepackage{fullpage}
\usepackage{eso-pic}

\setlength{\unitlength}{1cm}

%Paramètres de langue
\setdefaultlanguage{english}
\setotherlanguage{french}

% Marges du document
\usepackage[lmargin=3cm, rmargin=2cm, vmargin=2.5cm]{geometry}

% En-tête et pieds de page
\iffalse
\usepackage{fancyhdr}
\pagestyle{fancy}
\fancyhead[R]{\textsc{sous-titre}}
\fancyhead[L]{Paradis Enzo}
\renewcommand{\footrulewidth}{1pt}
\fancyfoot[L]{catégorie}
\fancyfoot[R]{\thepage}
\fancyfoot[C]{}
\setlength{\headheight}{15pt}
\fi

%Packages maths
\usepackage{amsmath,amsfonts,amssymb,amsthm}
\usepackage{unicode-math}
%Autres
\usepackage{graphicx} %Insertion d'images
\usepackage{array}    %mise en forme tableau
\usepackage{hyperref} %liens internes au document
\usepackage{lipsum}   %lipsum
\usepackage{caption}  %caption sans figure
\usepackage{xcolor}
\usepackage{enumitem}
\usepackage{multicol}
\usepackage{subcaption}
\usepackage{tikz}
\usepackage{bbold}
\usepackage{appendix}
\usepackage{booktabs}
%code
\usepackage{minted}

%Optimisation
\renewcommand{\arraystretch}{1.2}
\renewcommand{\thesection}{\Roman{section}}

\numberwithin{equation}{section} 

\newcommand{\HRule}{\rule{\linewidth}{0.5mm}}
\newcommand{\blap}[1]{\vbox to 0pt{#1\vss}}
\newcommand{\py}[1]{\mintinline{python}{#1}}

\newcommand\PlaceFigure[3]{%
  \put(\LenToUnit{\dimexpr\paperwidth-#1},\LenToUnit{\dimexpr\paperheight-#2}){\blap{\llap{#3}}}%
}

\newcommand{\maketitlepage}{
\begin{titlepage}

% Add figure to titlepage
%   \AddToShipoutPicture{
%     }
 
    \begin{center}
        \huge\textbf{The Kuramoto Model}\\
        \large\textbf{Technical report for python numerical project}\\
        \vspace*{0.5cm}
        \large{Paradis Enzo}\\
        Student at the university of Bourgogne Franche-Comté\\
        Master CompuPhys - $1^{st}$ year\\
        \vspace*{1cm}
% You can add text here

    \end{center}
  
 
\end{titlepage}
\ClearShipoutPicture
\newpage}

\DeclareMathOperator{\e}{e}
\renewcommand{\exp}[1]{\e^{#1}}
\renewcommand{\vec}[1]{\overrightarrow{#1}}
\newcommand{\deriv}[1]{\mathrm{d}#1\\}
\newcommand{\moy}[1]{\ensuremath{\langle\;#1\;\rangle}\xspace}
\newcommand{\real}{\mbox{I\hspace{-.15em}R}}
\newcommand{\intg}{\mbox{I\hspace{-.15em}N}}
\newcommand{\oone}{\mbox{I\hspace{-.60em}1}}
\newcommand{\Ha}{\mathcal{H}}
\newcommand{\colVec}[4]{
    \begin{pmatrix} 
      #1\\ 
      #2\\
      #3\\
      #4
    \end{pmatrix}}
\newcommand{\rawVec}[4]{
    \begin{pmatrix} 
      #1 & #2 & #3 & #4
    \end{pmatrix}}
\newcommand*{\calVec}[4]{ 
    \left\lvert
      \begin{matrix} 
        #1\\
        #2\\
        #3\\
        #4
      \end{matrix}  
    \right.
  }
\newcommand{\derive}[2]{\dfrac{\deriv{#1}}{\deriv{#2}}}
\newcommand{\partd}[2]{\dfrac{\partial #1}{\partial #2}}
\newcommand{\ket}[1]{\ensuremath{|#1\rangle}\xspace}
\newcommand{\bra}[1]{\ensuremath{\langle #1|}\xspace}
\newcommand{\braket}[2]{\ensuremath{\langle #1 | #2 \rangle}\xspace}
\newcommand{\Braket}[3]{\ensuremath{\bra{#1}#2\ket{#3}}\xspace}
\newcommand{\abs}[1]{\ensuremath{\left|#1\right|}\xspace}
  %
\definecolor{couleur_lien}{RGB}{0, 102, 204}
\definecolor{couleur_lien2}{RGB}{0, 0, 254}
\hypersetup{
	colorlinks=true,
    linkcolor={couleur_lien2},
    citecolor={couleur_lien2},
    urlcolor={couleur_lien2}}

\unimathsetup{math-style=TeX}
\setmathfont{Libertinus Math}

